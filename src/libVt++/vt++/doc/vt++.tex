\documentclass{article}
\usepackage{verbatim,color}
\usepackage{graphicx}

\newcommand{\tcb}[1]{\textcolor{blue}{#1}}
\newenvironment{code}[1]{\color{#1}\verbatim}{\endverbatim\color{black}}

\title{Vt++ Version 1.3}
%\\[1ex]\normalsize{HERA-B Note 00-175}
\author{Thorsten Glebe\\
        Max-Planck-Institut f\"ur Kernphysik, Heidelberg\\
        T.Glebe@mpi-hd.mpg.de\\[1ex]
        http://www.mpi-hd.mpg.de/herab/vt++}

\begin{document}
\maketitle
\abstract{Vt++ is the C++ reimplementation of the core part of the FORTRAN
  based vertex fitting package \textbf{Vt}, written by T. Lohse. It consists
  of the Kalman filter based vertex fit and the inverse Kalman filter. This
  part of \textbf{Vt} is well investigated, used for many years and known to be
  correct.
  The goal is to provide an easy-to-use vertexing class which can be used
  to explore different vertexing strategies and to cure limitations of the 
  FORTRAN implementation.
  It is not the goal to provide 'the' final primary/secodary vertex finder,
  this is done by other vertexing software.}
\tableofcontents
%
% Section
%
\section{Introduction}
Vt, the FORTRAN based vertex fitting package written by T. Lohse
\cite{Vt}, has been used in HERA-B for more than five years and proven
to be a very useful tool. Nevertheless, some disadvantages are known,
too:
\begin{itemize}
\item As a consequence of the FORTRAN implementation, fixed buffer sizes for
  tracks and vertices exist, which can lead to problems while running on real
  data.
\item Support of Vt is missing.
\item The implementation has not been optimized with respect to execution speed.
\item For the usage of Vt in C++ some interfaces have been written, which, however,
  cannot overcome the above mentioned Vt-limitations. Therefore it is not
  attractive to use Vt in C++ code.
\end{itemize}
From this it became clear that a C++ port of (at least the base part) of Vt
would be very useful. It would make the interfaces obsolete and allow to cure
the disadvantages of the FORTRAN implementation. Making use of state-of-the-art
C++ coding and STL also allows to write a code which is much better to maintain
and to extend.  Porting Vt to C++ is easy and fast, as one can always compare
with the original code.

Vt++ consists at the moment of the Kalman filter based vertex fit, the inverse
Kalman filter (for track removal), the mass constrained vertex fit and some
distance functions. Three simple analytical vertex computation routines are
available, too. The classes can be used to write more complex data analysis
routines. The documentation about Vt++ version 1.0 has been published as a
HERA-B note \cite{Vt++}.

The Vt++ key features are:
\begin{itemize}
\item Dynamic memory is used to store tracks and vertices - no limitations on
  the number of tracks or vertices does exist.
\item Selecting tracks for vertexing is under user control. Vt++ can work with
  track objects which can be filled from RTRA, RSEG and MTRA.
\item Vertex fits are not saved automatically in the RVER table - the user can
  decide which vertices should be kept.
\item Kalman filter computation is based on a Matrix package which is
  independent from Vt++ and ARTE.
\item Removable track-vertex object relations allow fast track-vertex
  assignment/removal.
\item Three additional simple algorithms for vertex calculation are provided:
  \begin{enumerate}
    \item unweighted 2-D minimal distance
    \item weighted 2-D minimal distance
    \item unweighted 3-D minimal distance
  \end{enumerate}
\item A vertex object is a STL container for track-vertex relations. A
  STL-iterator is provided to loop over the relations (see examples in section
  \ref{IteratorEx}).
\end{itemize}
%
% Section
%
\section{Compilation and Versioning}
From ARTE-03-07-r6 on Vt++ will be part of the ARTE distribution. In case
you want to recompile the package, please note the following compile-flags:
\begin{description}
\item[-DVtDEBUG] produces extensive debugging output.
\item[-DVtFAST]  this compilation flag removes the \texttt{assert()} macros
  from the MATRIX package. This makes the code faster but also less safe. One
  should use this flag only if it is verified that MATRIX computation is
  correct. The assert statements are Heisenbug-free\footnote{A Heisenbug is a
    bug which disappears when the debugger is turned on. In case assert-macros
    generate side-effects they are a source of those bugs.}.
\end{description}
The Vt++ package has an individual versioning independent from the
ARTE version number. The Vt++ version number plus some additional
information is represented by the \texttt{\tcb{VtIni}} class and
printed out when the VtIni object file is loaded into memory.
In case of ARTE the version string is the first information which is
printed by the ARTE job. For more information see section \ref{VtIni}.
%
% Section
%
\section{Speed}
Up to now, Vt++ has not been optimized with respect to execution
speed. In the current phase of development it is more important to
assure correctness, flexibility and maintainability. The structure of
the code will allow a detailed profiling such that considerable
improvements in execution speed will be possible. See also \cite{FastC++}.
A speedup factor in the range of 2-15 can be expected \cite{ExprT}.
%
% Section
%
\section{Vertexing}
%
%
\subsection{Classes}
Vt++ consists of the following classes:
\begin{description}
\item[Vertex] the vertexing class. A vertex can be constructed from
  \textbf{RVER}, \textbf{RTRA}, \textbf{MTRA}, \textbf{RSEG}, \textbf{Track}
  and \textbf{Vertex} objects. Tracks can be added via the \textbf{push\_back}
  member function. A vertex can own tracks or use tracks from an external list
  of track objects. The latter is the optimal solution in case of pattern
  recognition algorithms. A Vertex object also represents the incoming
  track. In case you compute the track parameters of the incoming track, you
  can use Vertex objects again as an input for vertexing routines (see example \ref{Mother}).
\item[Track]  the track representation class. A track object can be
  constructed from \textbf{RTRA}, \textbf{MTRA}, \textbf{RSEG} and
  \textbf{Track} objects. It stores the track parameters, the ARTE-pointers and
  the Track-Vertex relation. In case of a global track list this allows to loop
  over all vertices a track is assigned to. Relations are optimized
  with respect to fast addition and removal of relations. Therefore no
  direct access to a certain relation is possible. An emulation of
  this behaviour is provided, but one has to keep in mind that this is
  in most cases less efficient.
\item[Kalman] the Kalman mathematics class. A Kalman object is constructed by a
  Relation and consists of all relevant Kalman filter objects (matrices and
  vectors) of a track belonging to a certain vertex. For each Track-Vertex
  relation a new Kalman object is generated automatically. As the Kalman filter
  information of all filter steps is saved in this way, track removal from a
  vertex is a fast operation in the Kalman language.
\item[Relation] the Track-Vertex-Kalman relation class. A Relation is
  constructed by a Track and a Vertex object. It consists of the
  references to a Track, a Vertex and a Kalman object.  As FORTRAN
  doesn't know dynamic memory allocation, removable object relations
  are extremely difficult to program in FORTRAN. As a consequence it
  is not possible to remove objects from ARTE-tables as then relations
  would become invalid. Vt++ is an example which shows the
  implementation of a removable object relation in C++ (see example
  \ref{ex5.11}).  Vt++ is an application with which this idea can be
  further developed and which can then be applied to other software
  (e.g. CLUE).
\item[RelationList] This is a container class for pointers to Relation
  objects. As the Track and the Vertex classes are inherited from RelationList,
  the many-to-many relation between them is automatically established. This
  class implements the STL compatible iterators which are needed to
  loop over the relations in a convienient and standardized way.
\item[MassC] the mass constraint class. A MassC object describes a single mass
  constraint. This class keeps track of the tracks which participate in a mass
  constraint. A Vertex object can have many mass constraints. For an example,
  see section \ref{MassConstr}.
\end{description}
%
%
\subsection{Namespace}
Vt++ is coded in the \textbf{VERTEX} namespace. Therefore one can
abbreviate the class names by declaring the namespace like this:
\begin{code}{blue}
  using namespace VERTEX;
\end{code}
Otherwise one has to use the full classname, which means to specify the
\textbf{VERTEX::} prefix.
%
%
\subsection{Functions and Templates}
Some global functions and templates are provided, too. They belong to the
\textbf{VERTEX} namespace. Functions which are not specific for Vt++ can be
found in the \textbf{VtUtil.hh} header file. The following functions are
implemented:
\begin{description}
\item[sqr] template to compute the square of a number: $\textrm{sqr}(x) = x\times x$
\item[sgn] template to compute the sign of a number: $\textrm{sgn}(x)
  = \left\{\begin{array}{rc} 0 & x = 0\\ -1 & x < 0\\ 1 & x > 0\\\end{array}\right.$
\item[getIDX] template to compute the index of an ArtePointer:
  \begin{code}{blue}
    ArtePointer<RSEG> rseg(5);
    unsigned int idx = getIDX<RSEG>(rseg);
  \end{code}
\item[getPTR] template to compute the ArtePointer from an index:
  \begin{code}{blue}
    ArtePointer<RSEG> = getPTR<RSEG>(idx);
  \end{code}
\end{description}
The \textbf{VtDistance.hh} headerfile declares distance functions (part of
namespace \textbf{VERTEX}). The most important are:
\begin{description}
\item[distanceChi2(const Track\& t1, const Track\& t2)] $\chi^2$ distance between
  two Track objects.
\item[distance(const Track\& t, const Vertex\& v)] spatial distance between a
  Track and a Vertex object.
\item[distanceChi2(const Vertex\& v, double x, double y, double z)] $\chi^2$
  distance beween a Vertex object and a space point.
\item[distanceChi2(const Vertex\& v1, const Vertex\& v2)] $\chi^2$ distance between
  two vertices.
\item[Vertex::distance(const Track\& t)] $\chi^2$ distance between
  track and vertex (only available as member function of the Vertex class).
\end{description}
Some more useful distance functions have been added. Please see the reference
manual for a complete list of distance functions.
%
% Section
%
\section{Matrix computation}
A Kalman filter mainly consists of matrix operations. To exploit the
abilities of C++ it is natural to base the Kalman filter computation
on a Matrix class. Several packages with matrix computation abilities
exist: CERNLIB, ROOT, LAPACK, CLHEP. The disadvantage of those
packages is that they introduce a lot of overhead as they come with a
lot of (unwanted) stuff, are difficult to compile (if the needed
binary type is not available) and difficult to understand. In Vt++, the goal
was to have a matrix computation package which contains only the most
needed operations, is easy to oversee and simple to adopt to own
needs. The matrix classes are designed to call CERNLIB functions. In
case one wants to switch to other products (like eg. LAPACK), this is easy
to achive due to the modularity of the code.
%
%
\subsection{Classes}
Vt-Matrix consists of the following classes:
\begin{description}
\item[VtVector] simple vector class. It can be constructed from a
  dimension $n$, another VtVector object or from a couple of double
  precision numbers (between 2 and 6 constructor arguments). Operators
  \texttt{=}, \texttt{+=}, \texttt{-=}, \texttt{*=}, \texttt{/=}, \texttt{+},
  \texttt{-}, \texttt{*} (scalar product) are defined.
\item[VtMatrix] simple $m\times n$ Matrix class which can be
  constructed from a VtMatrix object or by specifying the dimensions
  $m$, $n$. The matrix elements can be accessed via the bracket
  operators:
  \begin{code}{blue}
    VtMatrix a(2,3);  // 2x3 matrix
    a(0,0)  = -2.5;
    a[0][1] = 5.2;
    cout << a;
  \end{code}
  The first access method is more efficient than the second.
  The following operations are defined: \texttt{=}, \texttt{+=}, \texttt{-=},
  \texttt{*=}, \texttt{/=}, \texttt{+}, \texttt{-} (both unary and binary),
  \texttt{*}. Transponation is provided via a member function:
  \begin{code}{blue}
    const VtMatrix b = a.T(); // a is not modified
    a.VtT(); // transform a into it's transpose
  \end{code}
\item[VtSqMatrix] representation of square $n\times n$
  matrices. This class is inherited from VtMatrix. Additional
  functionality is the computation of the inverse (via CERNLIB dinv()
  function) and the determinant (via CERNLIB dfact() function). Two
  different member functions for matrix inversion exist, one for
  returning the inverse without modifying the matrix object, another
  one for transforming the matrix object into it's inverse.
\item[VtSymMatrix] representation of symmetric matrices. This class is
  inherited from VtSqMatrix. Operators \texttt{=}, \texttt{+}, \texttt{-},
  \texttt{*}, \texttt{+=}, \texttt{-=}, \texttt{*=}, \texttt{/=} are optimized
  by exploiting the symmetry. The member function
  \texttt{\tcb{product(const VtVector\& v)}} allows to compute the product
  $v^t*A*v$ using an optimized expression. Matrix inversion is done via CERNLIB
  function dsinv().
\item[CMatrix] covariance matrix class specific to HERA-B. This class
  is derived from VtSymMatrix and can be constructed from RTRA, RSEG, MTRA or
  another VtSymMatrix, VtSqMatrix, VtMatrix or CMatrix object. The CMatrix
  dimension is 5. \textbf{Note:} the CMatrix class makes the transformation
  $\sigma(Q/p) \rightarrow \sigma(p)$ ($Q$ is the charge) because $\sigma(p)$
  is needed by Vt. The \texttt{\tcb{propagate(double z)}} member function
  allows to propagate the covariance matrix to a given z-position. Each element
  of the covariance matrix can be accessed by a dedicated member function:
  \begin{code}{blue}
    ArtePointer<RTRA> rtra(5);
    CMatrix cov(rtra);
    cout << " rtra->cvf[0]: " << cov.x() 
         << " rtra->cvf[1]: " << cov.x_y()
         << " rtra->cvf[2]: " << cov.x_tx()
         << endl;
  \end{code}
  Matrix elements can be modified in the following way:
  \begin{code}{blue}
    cov.set_x_y(0.00034);
    cov.set_x_tx(1e-4);
  \end{code}
  To get a propagated matrix element:
  \begin{code}{blue}
    double dz = -5.;
    double new_x = cov.x_prop(dz);
    double new_x_tx = cov.x_tx_prop(dz);
  \end{code}
  For a full list of access methods see the reference manual.
\item[VtNegMatrix] class for negative matrices. This class is inherited
  from VtMatrix. It optimizes computations like $C=-A*B$ where the
  negative of a matrix is involved.
  A FORTRAN implementation would look like this:
  \begin{code}{blue}
    do  i = 1, 3
      do  j = 1, 3
        C(i,j) = 0.
        do  m = 1, 3
          C(i,j) = C(i,j) - A(i,m)*B(m,j)
        enddo
      enddo
    enddo
  \end{code}
  Using the VtMatrix class, the computation would look like this:
  \begin{code}{blue}
    for(unsigned int i=0; i<3; ++i)
      for(unsigned int j=0; j<3; ++j)
          A(i,j) *= -1;

    for(unsigned int i=0; i<3; ++i)
      for(unsigned int j=0; j<3; ++j) {
        C(i,j) = 0;
        for(unsigned int m=0; m<3; ++m)
          C(i,j) += A(i,m)*B(m,j);
      }
  \end{code}
  With the VtNegMatrix class, computation can be as efficient as in the
  FORTRAN example and as compact as the notation $C=-A*B$.
\end{description}
%
%
\subsection{Namespace}
Vt-Matrix is coded in the \textbf{MATRIX} namespace. Therefore one can
abbreviate the class names by declaring the namespace like this:
\begin{code}{blue}
  using namespace MATRIX;
\end{code}
Otherwise one has to use the full classname, which means to specify the
\textbf{MATRIX::} prefix.
%
% Section
%
\section{VERTEX Examples}
Some examples on how to use Vt++ are presented in the following subsections. A
full list of available member functions can be found in the Vt++ reference
manual. In this section the symbols \texttt{r1}, \texttt{r2}, \ldots represent
RTRA, RSEG or MTRA ArtePointers.
%
%
\subsection{Using the header file}
In order to make use of Vt++, one has to include the appropriate headerfile:
\begin{code}{blue}
  #include "vt++/Vt++.hh"
\end{code}
%
%
\subsection{Version information}
\label{VtIni}
Information about the Vt++ package can be obtained via the
\texttt{\tcb{VtIni}} class:
\begin{code}{blue}
  VERTEX::VtIni pkg;
  cout << pkg << endl;
\end{code}
A string is printed which contains the name of the package, the
version number and the creation time of this version. Example:
\begin{code}{blue}
vt++ version /1-2 loaded (Mit Okt 25 18:29:27 MEST 2000).
\end{code}
It is also possible to get the information separatly:
\begin{code}{blue}
  VERTEX::VtIni pkg;
  cout << "Package name: "  << pkg.name()
       << "Version: "       << pkg.version()
       << "Creation time: " << pkg.date()
       << endl;
\end{code}
%
%
\subsection{Example of two track vertices}
A complete example on how to use the Vertex class to fit two-track
vertices is shown here:
\begin{code}{blue}
  #include "vt++/Vt++.hh"
  ...

  for(ArteTable<RTRA>::const_iterator 
        it1  = ArteTable<RTRA>::begin(); 
        it1 != ArteTable<RTRA>::end();
      ++it1) {
      ArtePointer<RTRA> r1 = *it1;

      if(r1->hsi < 6) continue;  // simple track selection

      ArteTable<RTRA>::const_iterator tmp = it1;
      for(ArteTable<RTRA>::const_iterator it2 = ++tmp;
          it2 != ArteTable<RTRA>::end(); ++it2) {
          ArtePointer<RTRA> r2 = *it2;
      
         if(r2->hsi < 6) continue;  // simple track selection

          VERTEX::Vertex v(r1,r2);
          if(v.findVertexVt() == true && v.chi2() < 4.) {
            cout << "vertex found: " << v << endl;
          }
      }
  }
\end{code}
%
%
\subsection{Vertex fit methods}
Five different methods to compute a vertex are available:
\begin{code}{blue}
   VERTEX::Vertex v(r1,r2);
   v.findVertexVt();     // Vt Kalman filter method
   v.VtMass();           // Vt mass constrained Kalman filter
   v.findVertex2D();     // unweighted 2-D min. distance
   v.findVertex3D();     // unweighted 3-D min. distance
   v.VtEstimateVertex(); // weighted 2-D min. distance
\end{code}
\textbf{Note:} the functions return \texttt{true} in case of success, else \texttt{false}.
For mass constrained fits, see also section \ref{MassConstr}.
%
%
\subsection{Using momentum in the vertex fit}
Per default, the Kalman filter uses the track momentum for the vertex fit. To
make a vertex fit without momentum (straight line fit) do the following:
\begin{code}{blue}
  VERTEX::Vertex v(r1,r2);
  v.use_momentum(false);
  v.findVertexVt();
\end{code}
To turn on momentum usage again, type
\begin{code}{blue}
  v.use_momentum(true);
  v.findVertexVt(); // vertex fit with momentum
\end{code}
You can also set the momentum usage for tracks individually:
\begin{code}{blue}
  ArtePointer<RTRA> r1 = ...;
  ArtePointer<RTRA> r2 = ...;
  VERTEX::Vertex v(r1,r2);

  // turn off momentum usage for tracks without OTR segment!
  for(RelationList::iterator it = v.begin(); it != v.end();
      ++it) {
     // get back ArtePointer from Track object
     ArtePointer<RTRA> rtra = it->track.rtra();

     if(rtra->hou == 0)
       it->kalman.use_momentum(false);
  }

  if(v.findVertexVt() == true) {
    cout << "Vertex found: " << v << endl;
  }
\end{code}
For more information about the Vertex-Iterator see section \ref{IteratorEx}.

\textbf{Note:} The other three vertexing routines do not use momentum information.
%
%
\subsection{Adding more Tracks}
Because of performance reasons, a vertex is not automatically refitted
if a track is added. The user can decide when the vertex fit should be done:
\begin{code}{blue}
  VERTEX::Vertex v(r1,r2);
  v.push_back(r3);
  v.push_back(r4);
  v.findVertexVt();
  if(v.valid()) cout << v << endl;
\end{code}
To use an efficient way to decide wether a track is compatible with
the vertex or not, one can calculate the $\chi^2$-distance between the
track and the vertex:
\begin{code}{blue}
  VERTEX::Vertex v(r1,r2);
  if(v.findVertexVt() == true) {
    Track t(r3);

    // add track if chi2-dist is < 4. (ndf = 2)
    if( v.distance(t) < 4.) {
      v.push_back(t);
      v.findVertexVt();
    }
  }
\end{code}
%
%
\subsection{Vertex members functions}
Vertex information can be accessed like this:
\begin{code}{blue}
  VERTEX::Vertex v(r1,r2);
  if(v.findVertexVt() == false) return;

  cout << " x: " << v.vx()
       << " y: " << v.vy()
       << " z: " << v.vz()
       << " error on x: " << v.vxerr()  // sqrt(rver->cv[0])
       << " error on y: " << v.vyerr()  // sqrt(rver->cv[2])
       << " error on z: " << v.vzerr()  // sqrt(rver->cv[5])
       << " chi2: " << v.chi2() // only available for Kalman-fit
       << " no of tracks: " << v.ntracks()
       << endl;
\end{code}
See the reference manual for a full list of access functions.
%
%
\subsection{Mean distance and opening angle}
To compute the mean distance of the tracks to the vertex and the rms opening
angle, a special member function exists:
\begin{code}{blue}
  VERTEX::Vertex v(r1,r2);
  v.push_back(r3);
  if(v.findVertexVt()) {
    v.rmsDistAngle();
    cout << " rms distance: " << v.dist()
         << " rms angle: "    << v.angle()
         << endl;
  }
\end{code}
%
%
\subsection{Reconstructed Mass}
To compute the reconstructed mass, one can choose to use the refitted momenta
(in case the vertex fit was done using the Kalman filter) or to use the
original track momenta. This behaviour can be toggled:
\begin{code}{blue}
  VERTEX::Vertex v(r1,r2);
  if(v.findVertexVt()==false) return;
  cout << "mass using orig. track momenta, particle rest mass=0: " 
       << v.mass() << endl;

  v.use_kalman(true); // turn on usage of refitted track params.

  cout << "mass using refitted momenta, particle rest mass=0: " 
       << v.mass() << endl;

  v.use_kalman(false); // turn off refitted track params.
\end{code}
There are several methods to assign a rest mass to the tracks on the vertex.
To assign a single rest mass to all tracks, do it like this:
\begin{code}{blue}
  VERTEX::Vertex v(r1,r2);
  if(v.findVertexVt() == true) {
    cout << " pi-pi mass: " << v.mass(0.1396)
         << " K-K mass: "   << v.mass(0.494)
         << endl;
  }
\end{code}
For vertices with up to 4 tracks, rest masses can be assigned to the individual
tracks:
\begin{code}{blue}
  VERTEX::Vertex v(r1,r2);
  if(v.findVertexVt()) {
    cout << " pi-K mass: " << v.mass(0.1396,0.494)
         << " K-pi mass: " << v.mass(0.494,0.1396)
         << endl;
  }
  v.push_back(r3);
  if(v.findVertexVt()) { // fit 3-track vertex
    cout << "pi-K-pi mass: " << v.mass(0.1396,0.494,0.1396)
         << endl;
  }
  // wrong: 3-track vertex but only 2 arguments:
  cout << " wrong mass assignment: " << v.mass(0.1396,0.494)
       << endl;
\end{code}
In case of a wrong mass assigment, the result is $-1$.
The mass assignment can also be done in case of $n$ tracks:
\begin{code}{blue}
  VERTEX::Vertex v(r1,r2);
  v.push_back(r3);
  v.push_back(r4);
  v.push_back(r5);
  v.push_back(r6);
  v.push_back(r7);

  // rest mass vector
  vector<double> m0(7);
  m0[0] = 0.1396*0.1396; // note: square of rest mass
  m0[1] = ...
  ...
  m0[6] = 0.494*0.494;

  cout << "mass: " << v.mass(m0);
\end{code}
You can also assign a rest mass permanently to a track:
\begin{code}{blue}
  ArtePointer<RTRA> r = ...;
  ArtePointer<TOPP> pi(101);

  Track t(r);
  Track u(r);

  t.rm(0.140);      // assign rest mass by hand
  u.rm(pi->mass);   // or via TOPP-table:

  cout << "Track rest mass: " << u.rm() << endl; 
\end{code}
In case you do not specify a rest mass, the permanent rest mass will be taken:
\begin{code}{blue}
  ArtePointer<TOPP> pi(101);

  Track t1(r1);
  Track t2(r2);
  t1.rm(pi->mass);    // set pion rest mass
  t2.rm(pi->mass);    // for both tracks
  VERTEX::Vertex v(t1,t2);
  
  if(v.findVertexVt() == true) {
    cout << "pi-pi mass: " << v.mass() << endl;
  }
\end{code}
%
%
\subsection{Using the Vertex-Iterator}
\label{IteratorEx}
To loop over the tracks assigned to a vertex and to access the Kalman filter
information, the Relation iterator can be used in the following way:
\begin{code}{blue}
  using namespace VERTEX;
  ...

  Vertex v(r1,r2);
  if( v.findVertexVt() == false ) return;
    
  for(const_iterator it = v.begin(); it != v.end(); ++it) {
    const Track& track = it->track;
    cout << "original track parameters:"
         << " x: " << track.x()
         << " y: " << track.y() 
         << " p: " << track.p()
         << endl;
  }
\end{code}
It is also possible to iterate over the tracks in reverse order:
\begin{code}{blue}
  using namespace VERTEX;
  ...

  Vertex v(r1,r2);
  if( v.findVertexVt() == false ) return;
    
  for(const_reverse_iterator it = v.rbegin(); it != v.rend();
      ++it) {
    const Track& track = it->track;
    cout << "original track parameters:"
         << " x: " << track.x()
         << " y: " << track.y() 
         << " p: " << track.p()
         << endl;
  }
\end{code}
If you want to have a direct access to the track or kalman information without
using the iterator, the following can be done:
\begin{code}{blue}
  using namespace VERTEX;
  ...

  Vertex v(r1,r2);
  if(v.findVertexVt()) {
    const Track& track1 = v[0]->track;
    const Track& track2 = v[1]->track;
    cout << " Track 1: " << track1 
         << " Track 2: " << track2 
         << endl;
    cout << " p1: "      << track1.p() 
         << " p2: "      << track2.p()
         << endl;
  }
\end{code}
This method also works with tracks:
\begin{code}{blue}
  Track t1(r1);  // generate Track objects from
  Track t2(r2);  // ArtePointers

  Vertex v(t1,t2);
  if(v.findVertexVt()) {
    const Vertex& vertex1 = t1[0]->vertex;
    const Vertex& vertex2 = t2[0]->vertex;
    // both vertex objects are the same in
    // this simple example...
  }
\end{code}
But be careful: \textbf{direct access is only emulated and therefore less efficient!}
The Relation iterators are STL\footnote{Standard Template Library, see e.g. 
  \texttt{http://www.sgi.com/Technology/STL}.}
  compatible, that means standard STL algorithms
can be used:
\begin{code}{blue}
  Vertex v(r1,r2);
  v.push_back(r3);
  cout << "No of tracks in vertex: " 
       << distance(v.begin(),v.end()) 
       << endl;
\end{code}
%
%
\subsection{Mass constrained vertex fits}
\label{MassConstr}
A mass constrained vertex fit can be used in case several tracks are known to
originate from a narrow mass state. By this the precision of the parameters
obtained from the vertex fit can be improved. In general several mass
constraints with different combinations of tracks can be assigned to a vertex.
In the most simple case, all tracks from a vertex participate in a single mass
constraint:
\begin{code}{blue}
  VERTEX::Vertex v(r1,r2);

  // TOPP entry of mother particle (K0)
  ArtePointer<TOPP> K0(160);

  // all tracks in vertex originate from a
  // K0 mass state
  v.MassConstr(K0->mass);
  
  // mass constrained vertex fit
  if( v.VtMass() == true ) {
    cout << "K0-vertex: " << v << endl;
  }
\end{code}
\textbf{Note:} performing a mass constrained vertex fit always implies that the
track momenta are used. Therefore \texttt{VtMass()} calls
\texttt{use\_momentum(true)} first. This will overwrite all
\texttt{use\_momentum} settings in the track objects.

The rest masses of the outgoing particles enter into the mass constraint, so in
case you know in the above example that \texttt{r1} and \texttt{r2} are pions,
you can set the pion rest mass before you perform the vertex fit:
\begin{code}{blue}
  VERTEX::Vertex v(r1,r2);

  ArtePointer<TOPP> pi(101);
  ArtePointer<TOPP> K0(160);

  for(iterator it = v.begin(); it != v.end(); ++it) {
    it->track.rm(pi->mass);
  }

  v.MassConstr(K0->mass);

  if( v.VtMass() == true ) {
    cout << "Vertex: " << v << endl;
  }
\end{code}
In case you want to apply a mass constraint to a subset of tracks, you have
to select for each mass constraint the tracks which belongs to it. The
user has to manipulate a MassC object, which is generated by the Vertex object:
\begin{code}{blue}
  VERTEX::Vertex v(r1,r2);
  v.push_back(r3);

  ArtePointer<TOPP> Jpsi(295);
  ArtePointer<TOPP> B0(246);

  // get a mass constraint object 
  // attention: always use a reference!
  MassC mC& = v.addMassConstr();

  // set mass of mother particle
  mC.mass(Jpsi->mass);

  // loop over tracks on vertex
  for(const_iterator it = v.begin(); it != v.end(); ++it) {
    const Track& t = it->track;

    // store relations to tracks which belongs
    // to the mass constraint in MassC object
    if( ... ) 
      mC.push_back(*it);
  }

  // apply a second mass constraint for all tracks
  v.MassConstr(B0->mass);

  if( v.VtMass() == true ) {
    cout << "B0 vertex: " << v << endl;
  }
\end{code}
To remove all mass constraints from a vertex, use the
\texttt{\tcb{clearMassConstr()}} member function (removing a single mass
constraint object is not supported):
\begin{code}{blue}
  VERTEX::Vertex v(r1,r2);
  
  v.MassConstr(0.496);
  v.VtMass();
  v.clearMassConstr(); // remove mass constraints
  cout << "no of mass constraints: "
       << v.nMassConstr()
       << endl;
\end{code}
\textbf{Note:} In case a track is deleted from a vertex with mass constraints,
make sure that the track was not part of a mass constraint. Otherwise you will
have an invalid mass constraint object. If you want to delete a track which is
part of a mass constraint, delete the mass constraints, too:
\begin{code}{blue}
  VERTEX::Vertex v(r1,r2);
  v.push_back(r3);

  ArtePointer<TOPP> B0(246);
  v.MassConstr(B0->mass);
  v.VtMass();

  ...

  // remove track with biggest chi2 contribution
  v.remove_worst(); 
  // remove mass constraints if defined
  if(v.nMassConstr() > 0)
    v.clearMassConstr();
\end{code}
%
%
\subsection{Refitted Track parameters}
After a successful Kalman filter based vertex fit, the track
parameters are refitted with the constraint that the tracks originate
in the vertex. You can get the refitted parameters like in the
following example:
\begin{code}{blue}
  using namespace VERTEX;
  ...

  Vertex v(r1,r2);
  if( v.findVertexVt() == true ) {
    
    cout << "Refitted track parameters: "
         << " x: " << v.vx()  // same
         << " y: " << v.vy()  // for all
         << " z: " << v.vz()  // tracks
         << endl;
    for(iterator it = v.begin(); it != v.end(); ++it) {
      const Kalman& kalman = it->kalman;
      cout << " tx: " << kalman.tx()
           << " ty: " << kalman.ty()
           << " p: "  << kalman.p()
           << endl;
    }

    // or:
    cout << "refitted parameters of 1st track:"
         << " x: "  << v.vx()
         << " y: "  << v.vy()
         << " tx: " << v[0]->kalman.tx()
         << " ty: " << v[0]->kalman.ty()
         << endl;
  }

\end{code}
%
%
\subsection{Store Vertex in RVER table}
A Vertex object can be stored as an RVER object in the following way:
\begin{code}{blue}
  VERTEX::Vertex v(r1,r2);

  if(v.findVertexVt() && v.chi2() < 4.) {
    cout << "Created new RVER entry: "
         << v.createRVER() 
         << endl;
  }
\end{code}
The \texttt{\tcb{createRVER()}} member function returns an ArtePointer to the RVER
object. If the Vertex object is not generated by RTRA tracks, the RREL
relations are not set and a warning is printed to stdout.
\textbf{Note:} in case of RSEG�s, no transformation into the global HERA-B coordinate
system is done.
%
%
\subsection{Using an external Track list}
\label{ex5.11}
For more complicated operations it is better to have a global track list to
work with. The track list can be e.g. generated by a track selection routine:
\begin{code}{blue}
  using namespace VERTEX;
  ...
  vector<Track> trvec;   // Track container

  for(ArteTable<RTRA>::iterator it=ArteTable<RTRA>::begin();
      it!=ArteTable<RTRA>::end(); ++it) {
      ArtePointer<RTRA> rtra = *it;

      if(rtra->hsi > 6 && rtra->hou > 20) // track selection
         trvec.push_back(Track(rtra));    // store track
  }
\end{code}
Now one can combine the selected tracks to fit e.g. two-track vertices:
\begin{code}{blue}
  vector<Vertex*> vtx;   // vertex container
  
  for(unsigned int i=0; i<trvec.size(); ++i) {
    for(unsigned int j=i+1; j<trvec.size(); ++j) {
      Vertex* v = new Vertex(trvec[i],trvec[j]);

      if(v->findVertexVt())
        vtx.push_back(v);
      else
        delete v;
    }
  }
\end{code}
The construction of a vertex generates track-vertex relations.  For
sake of execution speed, one has to generate the Vertex-objects on the
heap and store the Vertex-pointer in order to avoid useless
copying. Despite of this fact it is always a bad idea to store Vertex
objects in STL containers. For an explanation see \cite{STL}.  If the
Vertex object is constructed on the stack memory, the relations are
lost\footnote{Because they are deleted by the destructor of the
temporary vertex object which owns the relations.}:
\begin{code}{blue}
  vector<Vertex> vtx;   // store found vertices

  for(unsigned int i=0; i<trvec.size(); ++i) {
    for(unsigned int j=i+1; j<trvec.size(); ++j) {
      Vertex v(trvec[i],trvec[j]);

      if(v.findVertexVt())
        vtx.push_back(v); // inefficient copying!
    }
  }
  // !! The Track-Vertex relations are lost now!!
\end{code}
Once the vertices are generated, one can exploit the relations.
Looping over all vertices a track belongs to and plot the $\chi^2$
is done in the following way:
\begin{code}{blue}
  // loop over tracks from global track list
  for(unsigned int i=0; i<trvec.size(); ++i) {
    const Track& track = trvec[i];

    // loop over vertices a track belongs to
    for(const_iterator it = track.begin(); 
        it != track.end(); ++it)
        cout << " vertex-chi2 " 
             << it->vertex.chi2()
             << " refitted track momentum: "
             << it->kalman.p()
             << endl;
  }
\end{code}
Looping over all tracks of a vertex is done like this:
\begin{code}{blue}
  // loop over all vertices
  for(unsigned int i=0; i<vtx.size(); ++i) {
    const Vertex& vertex = *vtx[i];

    // loop over tracks on a vertex
    for(const_iterator it = vertex.begin(); 
        it != vertex.end(); ++it)
      cout << " track-momentum: " 
           << it->track.p()
           << " refitted track-momentum: "
           << it->kalman.p()
           << endl;
  }
\end{code}
If you delete a Track from the global Track list, the track gets automatically
removed from all vertices it belongs to. Those vertices which lost the track
are flagged invalid and can be refittet:
\begin{code}{blue}
  trvec.pop_back(); // delete the last track 
                    // from the global track list
  
  // do a garbage collection
  for(unsigned int i=0; i<vtx.size(); ++i) {
    Vertex* vertex = vtx[i];

    if(vertex->size() < 2) {
        delete vertex;
        vtx[i] = 0; // set NULL pointer in vector
        continue;
    }

    // refit vertices which lost a track
    if(vertex->valid() == false) 
       vertex->findVertexVt();
  }
\end{code}
%
%
\subsection{Reconstruct the incoming track}
\label{Mother}
Once a vertex has been found, the track parameters of the incoming track can be
reconstructed using the vertex space-point and the sum of the momenta of the
outgoing tracks. The user can choose when the incoming track should be
reconstructed by calling the \texttt{\tcb{calc\_mother()}} member function:
\begin{code}{blue}
  using namespace VERTEX;
  ...
  Vertex v(r1,r2);
  if(v.findVertexVt() == true) { // vertex found
    v.calc_mother();
    cout << v;
    // or using member functions:
    cout << "Reconstructed Track:"
         << " x: "  << v.x()
         << " y: "  << v.y()
         << " z: "  << v.z()
         << " tx: " << v.tx()
         << " ty: " << v.ty()
         << " p: "  << v.p()
  }
\end{code}
The covariance matrix of the incoming track can be computed seperately. Once
the covariance matrix is available, the vertex object can be used again for a
Kalman-filter vertex fit:
\begin{code}{blue}
  using namespace VERTEX;
  ...
  Vertex v(r1,r2);
  if(v.findVertexVt() == true) {
    // track parameters of incoming track
    v.calc_mother();  
    // covariance matrix of incoming track
    v.calc_mother_cov();
    cout << "Cov. matrix of incoming track: "
         << v.V();

    // vertex incoming track with an RTRA track r3:
    Track t(r3);
    Vertex vv(t,v);
    if(vv.findVertexVt() == true) {
      cout << " Vertex found: " << vv << endl;
    }
  }
\end{code}
%
%
\subsection{Removing tracks from a Vertex}
\label{TrackRm}
There are several ways to remove a track from a vertex. Special member
functions are provided in order to remove the last track from the
track list and the track with the worst $\chi^2$ contribution:
\begin{code}{blue}
  using namespace VERTEX;
  ...
  Vertex v(r1,r2);
  v.push_back(r3);
  v.findVertexVt();
  // remove r3 again using the inverse Kalman filter
  if(v.remove_last() == true)
    cout << "last track removed" << endl;

  v.push_back(r3);
  v.push_back(r4);
  v.findVertexVt();
  // remove track with biggest chi2 contribution using
  // the inverse Kalman filter:
  if(v.remove_worst() == true)
    cout << "track with largest chi2 removed" << endl;
\end{code}
As \texttt{\tcb{remove\_worst()}}, \texttt{\tcb{remove\_last()}} and
\texttt{\tcb{VtRemoveTrack()}} use the inverse Kalman filter, make sure that
the vertex is fitted by \texttt{\tcb{findVertexVt()}} before you call these
functions. You cannot use these functions in case of a 2-track vertex.

The most flexible way to remove one or several tracks from a vertex is to use
the \texttt{\tcb{erase()}} member function. This allows to use the STL way of
removing an element from the relation container. The following example
shows how to remove all tracks with a $\chi^2$-contribution larger than
1 from the vertex:
\begin{code}{blue}
  using namespace VERTEX;
  ...
  // take a primary vertex from RVER
  ArtePointer<RVER> prim_vtx = ...;
  Vertex v(prim_vtx);
  // get full kalman information for this vertex
  v.findVertexVt();

  double prchi2 = 0;
  // STL way to use erase()
  for(iterator it=v.begin(); it!=v.end();) {
    const Kalman& track = it->kalman;
    double chi2 = track->chi2();

    if( (chi2 - prchi2) > 1.) {
      it = v.erase(it);
    } else {
      ++it;
    }
    prchi2 = chi2;
  }

  // refit the vertex in case tracks were removed
  if(v.valid() == false)  
    v.findVertexVt();
\end{code}
The \texttt{\tcb{erase()}} member function removes the relations without
performing the inverse Kalman filter. It is under user control when
the vertex should be again refitted.
%
%
\subsection{Distance functions}
A couple of distance functions do exist. They are all defined in the VERTEX
namespace although not all distance functions deal with VERTEX classes:
\begin{code}{blue}
  using namespace VERTEX;
  ...
  ArtePointer<RVER> rver(4);
  ArtePointer<MVER> mver(3);
  ArtePointer<GTAR> gtar(2);

  // no clash with STL possible
  cout << "spatial rver-mver distance: " << distance(rver,mver) 
       << endl
       << "spatial rver-gtar distance: " << distance(rver,gtar)
       << endl;
\end{code}
Some distance functions are implemented as class members and can be 
called in two ways:
\begin{code}{blue}
  using namespace VERTEX;
  ...
  Vertex v1(r1,r2);
  Vertex v2(r3,r4);

  cout << "chi2 vertex-vertex distance: "
       << v1.distance(v2)
       << " or: "
       << distanceChi2(v1,v2) 
       << endl;

  cout << "chi2 distance vertex-space point: " 
       << v1.distance(0.1,0.1,-4.9)
       << " or: " 
       << distanceChi2(v1, 0.1, 0.1, -4.9) 
       << endl;
\end{code}
More special distance functions are named differently. E.g. the function
\texttt{\tcb{SGNdistance(const Track\& t, const Vertex\& v)}} returns the signed
spatial distance between Track and Vertex. The sign tells you if the point of
the track closest to the vertex is upstream ($sgn = -1$) or downstream ($sgn =
+1$) of the vertex. This information can be used for background suppression.
\subsection{Pull - distributions}
The pull-distributions of the reconstructed vertices in $x$, $y$ and $z$ are
shown in figure \ref{fig1}. A pull of a variable $x_i$ is defined like this:
\begin{equation}
  P(x_i) = \frac{x_i^{\textrm{REC}} - x_i^{\textrm{MC}}}{\sqrt{C_{ii}}}
\end{equation}
MC reconstruction was done with Geometry $99.0813$ and
ARTE-version $\textrm{03-07-r1}$. VDS track segments reconstructed by
CATS were used for the vertexing. One can see that apart from
non-gaussian tails the width of the pull-distributions is about 1 as one would
expect in case of a correct covariance matrix.
Using RTRA tracks for vertexing broadens the sigma of the
pull-distributions by about 30\%, which means that the errors in the RTRA
covariance matrix are underestimated.
\begin{figure}
\centering\includegraphics[width=\textwidth]{vtxpull_vt0813.epsi}
\caption{Vt++ vertex pull-distributions for $x$, $y$ and $z$ coordinate}
\label{fig1}
\end{figure}
%
% Section
%
\section{MATRIX Examples}
The vector and matrix objects which can be constructed use only
\texttt{\tcb{double}} precision numbers.
%
%
\subsection{VtVector Objects}
A default VtVector object can be constructed by giving the dimension as an
argument. The components will be initialized to 0.:
\begin{code}{blue}
  MATRIX::VtVector v(10);  // 10-dim vector
  cout << v << endl;
\end{code}
Another alternative is to define the vector by specifying the components
already in the constructor call:
\begin{code}{blue}
  MATRIX::VtVector v(1.0,2.0);  // 2-dim vector
  MATRIX::VtVector w(-2.4,3.14,2.714,4.55,17.77);
\end{code}
Up to 6 arguments are supported.

To access the vector components both \texttt{\tcb{operator()()}} and
\texttt{\tcb{operator[]()}} can be used:
\begin{code}{blue}
  MATRIX::VtVector v(4);

  v(0) = 5.1;  // or: v[0] = 5.1
  v(1) = -4.9; // or: v[1] = -4.9
  
  cout << "1st component: " << v[0]
       << "2nd component: " << v[1]
       << endl;
\end{code}
Both access methods are equivalent.

Mathematics with VtVector objects is intuitive:
\begin{code}{blue}
  MATRIX::VtVector v(1.0,2.0);
  MATRIX::VtVector w(3.0,4.0);

  // create vector via copy-constructor
  MATRIX::VtVector u = v + w;

  // use a temporary vector object
  cout << "v - w: " << v - w << endl;

  v += 3.;  // add 3 to each component
  double pi = 3.14159265;
  w *= pi;

  // skalar product is supported:
  double vw = v * w;
\end{code}
\textbf{Note:} The dimensions of the VtVector objects must be the same. If they
are not the same, the program will exit with an
\texttt{\tcb{assert}}-statement in case the code is compiled without the
\texttt{\tcb{-DVtFAST}} option. In case the \texttt{\tcb{-DVtFAST}} compile
option was used, the unequal vector dimensions are ignored, which might lead to
unexpected results.

The following special functions exist:
\begin{code}{blue}
  MATRIX::VtVector v(-2.5,4.77);

  cout << "dimension of v: " << v.size() // or v.nrow()
       << "vector length : " << v.mag()
       << "squared length: " << v.mag2()
       << endl;
\end{code}
%
%
\subsection{Constructing Matrix objects}
Four different kinds of matrices are supported: $n\times m$ matrices, $n\times
n$ matrices, symmetric $n\times n$ matrices and symmetric, positive
definite $5\times 5$ HERA-B Track covariance matrices.
For each of the above matrix types a class exists: \texttt{VtMatrix}, \texttt{VtSqMatrix},
\texttt{VtSymMatrix}, \texttt{CMatrix}. The constructors are different:
\begin{code}{blue}
  using namespace MATRIX;
  ...
  ArtePointer<RTRA> r(4);

  VtMatrix    A(2,3);  // construct a 2x3 matrix
  VtSqMatrix  B(2);    // construct a 2x2 matrix
  VtSymMatrix C(2);    // construct a 2x2 sym. matrix
  CMatrix     D(r);    // cov. matrix of 4th RTRA
\end{code}
For a full list of constructors see the reference manual.

To print a matrix object, just use \texttt{\tcb{cout}}:
\begin{code}{blue}
  MATRIX::VtSqMatrix A(2);
  A[0][0] = 0.5;

  cout << "Matrix A: " << A << endl;
\end{code}
%
%
\subsection{Accessing Matrix elements}
Matrix elements can be accessed in two ways: either using the
\texttt{\tcb{operator()()}} or the \texttt{\tcb{operator[]()}} member
functions:
\begin{code}{blue}
  using namespace MATRIX;
  ...
  VtMatrix A(2,3);

  A(0,0)  = 1.;
  A(0,1)  = 2.;
  A(1,0)  = 5.;
  ...
  // or:
  A[0][0] = 1.;
  A[0][1] = 2.;
  A[1][0] = 5.;
  ...
\end{code}
The \texttt{\tcb{operator()()}} access method is slightly more efficient, on
the other hand the \tcb{$[][]$}-method is well known from C and easier to recognize.
%
%
\subsection{Choosing the right Matrix class}
The special properties of the classes are exploited wherever possible, therefore
it is a matter of execution speed to choose the right matrix class.

Example: One can use the \texttt{VtMatrix}, \texttt{VtSqMatrix}
and \texttt{VtSymMatrix} class in order to represent a symmetric matrix, but
only the latter class is able to exploit the symmetry and produces faster
code. For the user the code looks in any case the same:
\begin{code}{blue}
  using namespace MATRIX:
  ...

  // inefficient symmetric matrix:
  VtMatrix A(2,2);
  A[0][0]           = 1.;
  A[0][1] = A[1][0] = 2.;
  A[1][1]           = 3.;

  VtMatrix    B = A;
  B *= 5.4;
  VtMatrix    C = B + A;

  // efficient symmetric matrix:
  VtSymMatrix a(2);
  a[0][0]           = 1.;
  a[0][1] = a[1][0] = 2.;
  a[1][1]           = 3.;

  VtSymMatrix b = a;
  b *= 2.7;
  VtSymMatrix c = b + a;
\end{code}
%
%
\subsection{Matrix operations}
Using the matrix classes for mathematical operations is more or less self
explaining:
\begin{code}{blue}
  VtMatrix A(2,3);

  // fill matrix A
  ...

  A *= 4; A += 3.14;

  VtMatrix B = A * 5.;
  VtMatrix C = -B + A;

  C = A.T();  // C is transpose of A
  C.VtT();    // transform C into it's transpose

  VtVector a(-1.0, 2.4, 3.5);
  VtVector b = A * a;
  cout << "scalar product: " << a * b << endl;

  VtSymMatrix D(3);  // 3x3 matrix
  
  // fill matrix D
  ...

  VtSymMatrix E = D.dsinv(); // return inverse
  D.VtDsinv(); // transform D into it�s inverse
\end{code}
For a full list of supported operations see the reference manual.

Special member functions exist to calculate the product $v^t*A*v$ and 
$B^t*A*B$ ($A$, $B$ are matrices, $v$ is a vector):
\begin{code}{blue}
  using namespace MATRIX;
  ...
  VtVector v(1.0,2.0,3.0);
  VtSymMatrix A(3);
  VtMatrix    B(3,2);

  // fill matrices A, B
  ...

  // calculate v^t*A*v:
  double vAv = A.product(v);
  
  // calculate B^t*A*B:
  VtSymMatrix C(2);
  C = A.product(B);
\end{code}
%
%
\subsection{Special member functions}
With the \texttt{\tcb{clear()}} member function you can set the
content of a matrix or vector object to zero:
\begin{code}{blue}
  using namespace MATRIX;
  ...
  VtVector a(2.5,1.0,0.9);
  a.clear();
  VtMatrix A(3,4);
  A = ...;
  A.clear();
\end{code}
To copy one matrix into another, two possibilities exist. In case
the dimensions of both matrices are equal, \texttt{\tcb{operator=()}} can
be used:
\begin{code}{blue}
  MATRIX::VtSymMatrix A(3);
  A = ...;
  MATRIX::VtSymMatrix B(3);
  B = A;  // operator=() is used
\end{code}
If the dimensions of the matrices are not equal, the \texttt{\tcb{copy()}} member function can be used:
\begin{code}{blue}
  MATRIX::VtMatrix A(2,4);
  A = ...;
  MATRIX::VtMatrix B(2,3);
  B.copy(A);
\end{code}
In this case only that part of \texttt{\tcb{A}} which fits into \texttt{\tcb{B}}
is copied.
Example:
\begin{code}{blue}
  A:   1.0  2.0  3.0  4.0    =>    B:   1.0  2.0  3.0
       5.0  6.0  7.0  8.0               5.0  6.0  7.0
\end{code}
With the \texttt{\tcb{place\_at()}} member function you can copy a
smaller matrix at a certain position in a bigger matrix. This is
useful in case you have to construct a big matrix out of several small matrices.
\begin{code}{blue}
  using namespace MATRIX;
  ...
  VtMatrix A(2,2);
  A = ...;
  VtMatrix B(5,5);
  B.place_at(A,2,1);
\end{code}
Example:
\begin{code}{blue}
  A:   1.  2.     =>  B:   0.  0.  0.  0.  0.
       3.  4.              0.  0.  0.  0.  0.
                           0.  1.  2.  0.  0.
                           0.  3.  4.  0.  0.
                           0.  0.  0.  0.  0.
\end{code}
%
%
%
\section{Vt++ running in ROOT/CINT}
It is possible to make use of Vt++ classes in ROOT\footnote{see
 \texttt{http://root.cern.ch}}. An indirect way is to precompile user
 functions which contain calls to Vt++ classes. But it is also
 possible to use Vt++ classes directly in CINT. To do so, it is
 neccessary to generate a shared library containing the Vt++
 classes. This can be done by modifying the
 \texttt{\tcb{Makefile.BEE}} makefile which is contained in the Vt++
 package. In future a Vt++ shared library will be provided with the
 BEE analysis software\footnote{see
 \texttt{http://www.mpi-hd.mpg.de/herab/clue}}.
For the following examples ROOT version 2.25.03 has been used.

Once the shared library (including the cint dictionary) is available,
it can be loaded into ROOT:
\begin{code}{blue}
  *******************************************
  *                                         *
  *        W E L C O M E  to  R O O T       *
  *                                         *
  *   Version   2.25/03   27 October 2000   *
  *                                         *
  *  You are welcome to visit our Web site  *
  *          http://root.cern.ch            *
  *                                         *
  *******************************************

CINT/ROOT C/C++ Interpreter version 5.14.50, Sep 2 2000
Type ? for help. Commands must be C++ statements.
Enclose multiple statements between { }.

root [0] gSystem->Load("libvt++.so");
vt++ version /1-2 loaded (Mit Okt 25 18:29:27 MEST 2000).
\end{code}
The version information will be printed automatically when the shared
library is loaded.

Now you can e.g. get the version information interactively:
\begin{code}{blue}
root [1] using namespace VERTEX
root [2] VtIni a
root [3] cout << a
vt++ version /1-2 loaded (Mit Okt 25 18:29:27 MEST 2000).
(class ostream)1076664544
\end{code}

In order to use the vertexing capabilities of Vt++, track information
is needed. This will be provided within the BEE package. Outside BEE,
the MATRIX classes can still be used:
\begin{code}{blue}
root [4] using namespace MATRIX
root [5] VtVector v(1,2,3)
root [6] cout << v

            1
            2
            3
(class ostream)1076664544
\end{code}
The classes can be used in the expected way:
\begin{code}{blue}
root [7] VtVector b(5,6,7)
root [8] cout << "scalar product: " << v * b << endl;
scalar product: 38
(class ostream)1076664544
\end{code}

A problem is the \texttt{\tcb{operator[]()}} in the matrix
classes. \texttt{\tcb{operator()()}} can be used instead:
\begin{code}{blue}
root [9] VtMatrix z(3,3)
root [10] z(0,1) = 1.
(const double)1.00000000000000000e+00
root [11] cout << z

            0             1             0
            0             0             0
            0             0             0
(class ostream)1076664544
\end{code}
%
% Bibliography
%
\begin{thebibliography}{99}
\bibitem{Vt} T. Lohse, Vertex Reconstruction And Fitting, HERA-B Note 95-013
\bibitem{Vt++} T. Glebe, Vt++ Version 1.0, HERA-B Note 00-175
\bibitem{FastC++} Todd L. Veldhuizen et al., Will C++ be faster than
  Fortran? Proceedings of the 1st International Scientific Computing
  in Object Oriented Parallel Environments (ISCOPE'97), 1997 Springer.
\bibitem{STL} D. K\"uhl, STL and OO Don't Easily Mix, Proceedings of the
  2000 Workshop on C++ Template Programming.
\bibitem{ExprT} T. Veldhuizen, Expression Templates, C++ Report, Vol. 7 No. 5
  June 1995.
\end{thebibliography}
%
% Section
%
%\section{ToDo}
%The following things are missing and might be added in future:
%\begin{itemize}
%\item Vertex fit using mass constraints.
%\item Decay chain fit using mass constraints.
%\item Target wire treatment.
%\item Add more useful member functions.
%\item Extend abilities of the MATRIX package.
%\item Optimize the code with respect to execution speed.
%\item Improve documentation, provide Web-page.
%\end{itemize}
\end{document}